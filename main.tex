%
% File naacl2019.tex
%
%% Based on the style files for ACL 2018 and NAACL 2018, which were
%% Based on the style files for ACL-2015, with some improvements
%%  taken from the NAACL-2016 style
%% Based on the style files for ACL-2014, which were, in turn,
%% based on ACL-2013, ACL-2012, ACL-2011, ACL-2010, ACL-IJCNLP-2009,
%% EACL-2009, IJCNLP-2008...
%% Based on the style files for EACL 2006 by 
%%e.agirre@ehu.es or Sergi.Balari@uab.es
%% and that of ACL 08 by Joakim Nivre and Noah Smith

\documentclass[11pt,a4paper]{article}
\usepackage[hyperref]{naaclhlt2019}

\usepackage{times}
\usepackage{latexsym}

\usepackage{tabularx}
\usepackage{amssymb}
\usepackage{amsmath}
\usepackage{latexsym}
\usepackage{enumitem}
\usepackage{booktabs}
\usepackage{url}
\usepackage{color} 
\usepackage{verbatim}
\usepackage{calc}
\usepackage{arydshln}
\setlength\dashlinedash{0.7pt}
\setlength\dashlinegap{1.pt}
\setlength\arrayrulewidth{0.3pt}
\usepackage{mathtools}
\usepackage[draft,textsize=tiny]{todonotes}
\newcommand{\raq}[1]{\textcolor{blue}{R: #1}}
\newcommand{\marco}[1]{\textcolor{red}{M: #1}}
\newcommand{\gbt}[1]{\textcolor{green}{G: #1}}
\newcommand{\new}[1]{\textcolor{red}{#1}}
\newcommand{\redd}{Reddit$_{13}$}
\usepackage{soul}

\aclfinalcopy % Uncomment this line for the final submission
%\def\aclpaperid{***} %  Enter the acl Paper ID here

%\setlength\titlebox{5cm}
% You can expand the titlebox if you need extra space
% to show all the authors. Please do not make the titlebox
% smaller than 5cm (the original size); we will check this
% in the camera-ready version and ask you to change it back.

\newcommand\BibTeX{B{\sc ib}\TeX}

\title{Short-term meaning shift: a distributional exploration}

\author{Marco Del Tredici$^*$ \ \ Raquel Fern\'andez$^*$ \ \  Gemma Boleda$^\dagger$\\
$^*$University of Amsterdam \ \ \ \ $^\dagger$Universitat Pompeu Fabra\\
  {\tt \{m.deltredici|raquel.fernandez\}@uva.nl}\\  
  {\tt gemma.boleda@upf.edu}
}

\date{}

\begin{document}
\maketitle
\begin{abstract}
We present the first exploration of meaning shift over short periods of time in online communities using distributional representations. We create a small annotated dataset and use it to assess the performance of a standard model for meaning shift detection on short-term meaning shift. We find that the model has problems distinguishing meaning shift from referential phenomena, and propose a measure of contextual variability to remedy this.
\end{abstract}

%============================
\section{Introduction}
\label{sect:Introduction}
Semantic change has received increasing attention in empirical Computational Linguistics / NLP in the last few years \cite{tang2018state,KutuzovEtal-coling2018}. Almost all studies so far have focused on meaning shift in long periods of time---decades to centuries. However, the genesis of meaning shift and the mechanisms that produce it operate at much shorter time spans, ranging
from the online agreement on words' meaning in dyadic interactions \cite{brennan1996conceptual} to the rapid spread of new meanings in relatively small communities of people \cite{wenger1998communities,eckert-mcconnellginet1992}.
This paper is, to the best of our knowledge, the first exploration of the latter phenomenon---which we call \textit{short-term meaning shift}---using distributional representations.

Short-term shift is usually hard to observe in standard language, such
as the language of books or news, which has been the focus of
long-term shift studies \cite[e.g.,][]{hamilton2016diachronic,kulkarni2015statistically}, since
it takes a long time for a new meaning to be widely accepted in the standard language. 
We therefore focus on the language produced in an online community of speakers, in which the 
adoption of new meanings happens at a much faster pace \cite{Clark96,hasan2009}.

Our contribution is twofold. First, we create a small dataset of short-term shift for analysis and evaluation, and qualitatively analyze the types of meaning shift we find.\footnote{Data and code are available at: 
\url{https://github.com/marcodel13/Short-term-meaning-shift}.
} This is necessary because, unlike studies of long-term shift, we cannot rely on material previously gathered by linguists or lexicographers.
Second, we test the behavior of a standard distributional model of semantic change when
 applied to short-term shift. 
Our results show that this model successfully detects most shifts in our data, but it overgeneralizes. Specifically, the model gets confused with contextual changes due to speakers in the community often talking about particular people and events, which are frequent on short time spans. 
We propose to use a measure of contextual variability to remedy this and showcase its potential to spot false positives of referential nature like these.
We thus make progress in understanding the nature of semantic shift and towards improving computational models thereof.

% \marco{but
% also that, on the one hand, a change in context of use - as measured
% by cosine distance between the vector representations for a word at different periods of time
% does not always correspond to a meaning shift and, on the other hand, a meaning shift can take place also without a change in context of use. These results challenge the Distributional Hypothesis itself, which is at the basis of distributional / word embedding approaches to language in Computational Linguistics.}\todo{note: R thinks this paragraph should be the last cause it is rhetorically more powerful than the one below.}

% \marco{Finally, we investigate the role of \textit{contextual variability}, i.e. the extent to which a word tends to occur in a large set of contexts, and show that it provides valuable information which is complementary to the one about cosine distance for the detection of meaning shift on the short term.} 

%but also overgeneralises to words that exhibit an increase in frequency but no meaning shift. This challenges the Distributional Hypothesis itself, which is at the basis of distributional / word embedding approaches to language in Computational Linguistics.

%\marco{moved the paragraph to the conclusion.}
%While preliminary, our investigation opens the way to a new line of
%research within diachronic studies in our field, focusing on
%Short-term Meaning Shift originated and spread in small communities
%of speakers. Besides being of intrinsic theoretical interest, such a
%phenomenon has practical implications for every NLP downstream task
%concerning online communities, as modeling how words change their
%meaning in each community should allow a better understanding of the
%contents produced by its members.

%we study something new: meaning shift in a (relatively) short span. Main contributions:
%manually annotated dataset and analysis of the performance of current models.

%*** Our goal is to assess the performance of NLP system for long term shift detection on a short period. 

%Things to stress: in the nlp field, large work has been carried out on long term meaning shift (actually, when talking about meaning shift, the long term one - 'gay' example - is the only one considered). However, there exists also another kind on of shift, which is on the short term. While the long term meaning shift is observed at the level of general lanaguage, meaning that the new meaning o a word is gradually spread at all the levels of a (national?) community of spekaers, short term changes can not be observed at the same level, because, by their nature, they originate and are shared by smaller communities of speakers. Only if they manage, at some point, to go beyond that community and spread in other, they can be modelled as long term shifts (in the same way of 'gay'). So, in a way, exploring short term changes means looking at the genesis of the ones investigated in the nlp field up to know.

%We need to say which are the kind of shifts observed, and for this we need some theoretical framework. In any case, we have for sure memes,  referential phenomena and some kind of metaphorical process. 

%we need to say that we want also to observe how model for long term meaning shift work on this dataset. 

%%% Local Variables:
%%% mode: plain-tex
%%% TeX-master: "main"
%%% End:

 
%============================
\section{Related Work}
\label{sect:Related_Work}
Several methods have been proposed to investigate long-term
meaning shift: common to all of them is the computation of
time-related distributional representations for words in the
vocabulary, and the sequential comparison of such representations in
order to detect a drop in self-similarity, usually
interpreted as a shift in meaning.

Among the most widely used techniques are Latent Semantic Analysis
\cite{sagi2011tracing,jatowt2014framework}, Topic Modeling
\cite{wijaya2011understanding,rohrdantz2011towards}, and simple
co-occurence matrices of target words and context terms 
\cite{gulordava2011distributional,xu2015computational}.
More recently, researchers have used word embeddings computed using the skip-gram model by \newcite{mikolov2013distributed}. Since embeddings computed in different semantic
spaces are not directly comparable, time related representation are usually made comparable either by aligning different semantic spaces 
%through a transformation matrix 
\cite{kulkarni2015statistically,azarbonyad2017words, hamilton2016diachronic} or by initializing the embeddings at $t$+1 using those computed at $t$ \cite{kim2014temporal, del2016tracing,phillips2017intrinsic,szymanski2017temporal}. We adopt the latter methodology.

Evaluation of semantic shift is difficult, due of the lack of annotated datasets \cite{frermann2016bayesian}. For this reason, evaluation is usually performed 
%qualitatively  by the investigators themselves
 by manually inspecting the $n$ words whose representation changes the most according to the model~\cite{hamilton2016diachronic,del2016tracing,kim2014temporal}. 
In this work, we introduce and make available a small dataset for
short-term meaning shift, which
allows for a more systematic evaluation and analysis and enables comparison in future studies.

%\raq{I find the last paragraph below a bit boring. The model is not our strong point, so no need to make this prominent. Instead I would highlight that previous work on LTMS is evaluated qualitatively or using judgements by the investigators. Instead, we use judgements by domain experts. }\todo{I agree with the change in emphasis, and also with Marco's remark that he'd like to keep this info. Marco, can you summarize and downplay the part about self-similarity and highlight what Raquel says?}
% ACL VERSION
%It is not straightforward to compute self-similarity across time with neural-based representations, since, due to the stochastic nature of the process, the embeddings computed in different semantic spaces are not directly comparable. Two main solutions have been proposed in the literature. The first consists in computing the embeddings separately for each time bin and then aligning different semantic spaces through a transformation matrix, that can be learned either employing linear regression \cite{kulkarni2015statistically,azarbonyad2017words} or orthogonal Procrustes \cite{hamilton2016diachronic}. The second type of solution,  introduced by \newcite{kim2014temporal}, is based on initializing new embeddings with existing embeddings. This methodology has been adopted in several recent diachronic studies\cite{del2016tracing,phillips2017intrinsic,szymanski2017temporal}, and we also adapt it for our purposes here.



%####################################################


%\gbt{Shouldn't you also cite your own previous work on meaning shift
%  in online communities?}

%Neural based work on diachronic meaning shift:\\
%\cite{kulkarni2015statistically,zhang2015omnia,hamilton2016diachronic}. stress
%that Hamilton says that emb are the best.\\ An alternative approach,
%which we also adopt - with a slight change - in our work, is
%introduced by \newcite{kim2014temporal}, who propose a simple but
%effective methodology to make vectors trained on different corpora
%directly comparable: embeddings created for year~$y$ are used to
%initialise the vectors for year $y+1$. The process is progressively
%applied to all time spans.\\ The SCAN model proposed by Frermann et
%al. (2016) is a dynamic Bayesian topic model.  EXPLANATOIN OF THE
%MODEL For each target word, a SCAN model is created that represents
%the meaning of the word with two distributions: a K-dimensional
%multinomial distribution over word senses φ and a V-dimensional
%distribution over the vocabulary ψ for each word sense. As a
%generative model, SCAN starts with a precision parameter κ , a Gamma
%distribution with parameters a and b that controls the extent of
%meaning change of φ. A three time step model involves the generation
%of three sense distributions φ , φ , and φ ,at times t-1, t and
%t+1. The sense distribution at each time generates a sense z, which
%in turn generates the context words for the target
%word. Simultaneously, for each time of t-1, t and t+1, the word-sense
%distributions ψ , , ψ , and ψ , are generated, each of which also
%generates the context words for the target word. The prior is an
%intrinsic Gaussian Markov Random Field which encourages smooth change
%of parameters at neighboring times. The inference is conducted with
%blocked Gibbs sampling.


%The phenomena of widening and narrowing are discussed in most
%studies, such as Sagi et al. (2009) and Gulordava et
%al. (2011). Novel senses, namely the birth of lexical senses are
%discussed in details in Erk (2006), Lau et al. (2012), Mitra et
%al. (2015), Frermann et al. (2016), Tang et al. (2016) and
%others. Pejoration and amelioration are dealt with in Jatowt et
%al. (2014)

%%% ******** FROM CLIC 16 PAPER ********
%The automatic modelling of diachronic shift of meaning has been investigated employing several different techniques. Among these, most recently, Latent Semantic Analysis \cite{sagi2011tracing,jatowt2014framework} and topic clustering \cite{wijaya2011understanding}.
%Vector representations for diachronic shift of meaning have been used by \newcite{gulordava2011distributional}, with a simple co-occurence matrix of target words and context terms. \newcite{xu2012computational} and \newcite{jatowt2014framework} experimented both with a bag-of-words approach and a more linguistically motivated representation that also captures %not only the frequency of co-occurring words but also 
%their relative positions in relation to the target word. 
%%who build 
%%context vectors for target words using simple co-occurrence matrix, where row elements are target words and column elements are context terms, and then apply local mutual information (LMI) to the matrix. Similar representations have been employed also by 
%
%Recently Word Embedding (see 4.4) have been used to investigate diachronic meaning shift: 
%%When using word embeddings \note{do we need to introduce the term?} to investigate diachronic meaning shift, 
%vectors are usually created independently for each time span and are mapped from one year to another via a transformation matrix, as %a transformation matrix is used to map vectors from one year to another %.  Indeed, even if there isn't an absolute correspondence between vectors in different semantic spaces, 
%the relative positions of vectors in different spaces are stable 
%\cite{kulkarni2015statistically,zhang2015omnia,hamilton2016diachronic}.
%%, \note{FIX: and thus a linear (Wt′  →t (w) ∈ Rd×d) that maps a word from φt to φt′ is feasible.}
%%These approaches are based on the idea that a transformation matrix can be learnt .... \note{mettere un po' di math: io direi solo spiegare in due parole} 
%%and also by \cite{}, who use the orthogonal Procrustes to align the learned embeddings of different periods.
%%%%%% THIS WAS A NICE BIT, BUT RELATED IS TOO LONG
%%A similar model is used by \newcite{zhang2015omnia}, who introduce the concept of ``global similarity'' as the correspondence across years between terms 
%%(``walkman'' and ``iPod'', e.g.)
% %that share similar contexts, found using a transformation matrix for vectors in different periods. 
%%In addition to this, they also compute a "local similarity", i.e. they also take into consideration the cosine similarity with some reference points in the synchronic semantic space.
%An alternative approach, which we also adopt - with a slight change - in our work, is introduced by \newcite{kim2014temporal}, who propose a simple but effective methodology to make vectors trained on different corpora directly comparable: embeddings created for year~$y$ are used to initialise the vectors for year $y+1$. The process is  progressively applied to all time spans.
%
%
%
%%\newcite{tahmasebi:15} identify three types of language change: the first one is spelling variation, the second one is what they name \textit{term to term evolution}, where different words are used to refer to the same concepts over time, including named entities. This class comprises cases such as ``The Great War'' and ``World War I'', or ``fine" which in the past was used in the same was as now is ``foolish". The third type they identify is \textit{word sense evolution}. 
%
%%In order to find word to word evolution, words are mapped to concepts (word senses), and then the context of concepts is compared. 
%
%
%%Instead, to find general word to word evolution, word senses must be used. A word is first mapped to a concept representing one of its word senses at one point in time. If one or several other words point to the same concept largely at the same period in time, then the words can be considered synonyms. If the time periods do not overlap, or only overlap in a shorter period, the words can be considered as temporal synonyms or word to word evolutions.
%
%%% see also: (context-based approach) Sagi, E., Kaufmann, S., Clark, B.: Semantic density analysis: comparing word meaning across time and phonetic space. In: Workshop on Geometrical Models of Natural Language Semantics, GEMS 2009, pp. 104–111 (2009). http://dl.acm.org/citation.cfm?id=1705415.1705429
%%% methods making use of probabilistic topic models:
%% Lau, J.H., Cook, P., McCarthy, D., Newman, D., Baldwin, T.: Word sense induction for novel sense detection. In: Conference of the European Chapter of the Association for Computational Linguistics, EACL 2012, pp. 591–601 (2012). http://aclweb.org/anthology-new/E/E12/E12-1060.pdf
%%% Wijaya, D.T., Yeniterzi, R.: Understanding semantic change of words over centuries. In: Workshop on DETecting and Exploiting Cultural diversiTy on the social web, DETECT 2011, pp. 35–40 (2011). doi:10.1145/2064448.2064475

%%% Local Variables:
%%% mode: latex
%%% TeX-master: "main"
%%% End:


%============================
\section{Experimental Setup}
\label{sec:setup}

% === DATA ==============

\paragraph{Data.}
We exploit user-generated language from an online forum of football fans,
namely, the r/LiverpoolFC subreddit, one of the many communities hosted
by the Reddit platform.\footnote{\url{https://www.reddit.com}.}
%\marco{I would remove this link, it is useless} 
%\raq{We can consider this at the end.}
We focus on a short period of eight years, between 2011 and 2017. 
In order to enable a clearer observation of short-term meaning shift, we define two
non-consecutive time bins: the first one ($t_1$) contains data from
2011--2013 and the second one ($t_2$) from 2017.\footnote{These choices
  ensure that the samples in these two time bins are approximately of the same size -- see Table~\ref{tab:data}. The
  r/LiverpoolFC subreddit exists since 2009, but very little content
  was produced in 2009--2010.}
 We also use a large sample of community-independent language for the
initialization of the word vectors, namely, a random crawl from Reddit
in 2013.\footnote{We used the Python package Praw for downloading the data, \url{https://pypi.python.org/pypi/praw}.} Table~\ref{tab:data} shows the size of each sample.

%----- TABLE ----------------------------------------------------
\begin{table}[t]\small
\centering
\begin{tabular}{lccc}
\bf sample & \bf time bin & \bf million tokens \\
 \hline
\redd &  2013 & {\raise.17ex\hbox{$\scriptstyle\sim$}}900 \\
LiverpoolFC$_{13}$ & 2011--13 & ~ 8.5\\
LiverpoolFC$_{17}$ & 2017 & 11.9\\ \hline
\end{tabular}
\caption{Time bin and size of the datasets.}
\label{tab:data}
\end{table}
%----- END TABLE -----------------------------------------



%=== MODEL ==========

\paragraph{Model.}
%We adapt the methodology introduced by 
%, which in turn builds on the skip-gram architecture by \newcite{mikolov2013distributed}
In the method proposed by \newcite{kim2014temporal}, word
embeddings for the first time bin $t_1$ are initialized randomly; then,
given a sequence of time-related samples, embeddings for $t_i$
are initialized using the embeddings of $t_{i-1}$ and further
updated. 
%with the standard skip-gram architecture.
If at $t_i$ the word is used in the same contexts as in $t_{i-1}$, its embedding will only be marginally updated, whereas a major change in the context of use will lead to a stronger update of the embedding. The model makes embeddings across time bins directly comparable.

%Recall that we want to spot changes that occur between 2013 and 2017 (the latter included). Since training directly on LiverpoolFC$_{13}$ is not possible due to data sparseness, 
We implement the following steps:
First, we create  randomly initialized word embeddings with the large sample \redd, to obtain meaning
representations that are community-independent.
%\footnote{\redd\ embeddings are initialized randomly.} \marco{Useless footnote, we say about the random initialization before}
We then use these embeddings
to initialize those in LiverpoolFC$_{13}$, update the vectors on this
sample, and thus obtain embeddings for time $t_1$. This step
adapts the general embeddings from \redd\  to the
LiverpoolFC community. Finally, we
initialize the word embeddings for LiverpoolFC$_{17}$ with those of $t_1$, train on this sample and get embeddings for $t_2$.\footnote{We implement the model
using the \texttt{gensim} library.}   
%To obtain word embeddings for time $t_2$, we
%initialize the word embeddings for LiverpoolFC$_{17}$ with those of $t_1$, 
%and train skip-gram further on the
%LiverpoolFC$_{17}$ data.

%We define the vocabulary to be represented as the intersection of the vocabularies of the three samples (\redd, LiverpoolFC$_{13}$, LiverpoolFC$_{17}$) which results in a vocabulary of 157k words.

The vocabulary is defined as the intersection of the
vocabularies of the three samples (\redd, LiverpoolFC$_{13}$,
LiverpoolFC$_{17}$), and includes 157k words.
%\todo{What is the vocabulary size?} 
%\redd~embeddings are initialized randomly. 
For \redd, we include only words that occur at least 20 times in the
sample, so as to ensure meaningful representations for each word,
while for the other two samples we do not use any frequency
threshold: Since the embeddings used for the initialization of
LiverpoolFC$_{13}$ encode community-independent meanings, if a word doesn't occur in
LiverpoolFC$_{13}$ its representation will simply be as in \redd,
which reflects the idea that if a word is not used in a community, then its meaning is not altered within
that community. 
%Instead, if a word's meaning changes in the
%community, then we expect the word embedding to change accordingly
%after training on the community-specific data.
%Analogous reasoning applies to LiverpoolFC$_{17}$. 
We train with standard skip-gram parameters \cite{levy2015improving}: window 5, learning rate 0.01, embedding dimension 200, hierarchical softmax.


%===== EVALUATION DATASET ===============


\paragraph{Evaluation dataset.}

We create a small dataset of words to be annotated by members of the r/LiverpoolFC
subreddit, that is, community members with domain knowledge (needed for this task) but no linguistic background.
We initially leverage information about increase in frequency, which has been shown to positively correlate with meaning change \cite{wijaya2011understanding,kulkarni2015statistically}, and sample content words with a significant increase in relative frequency (at least two standard deviations above the mean) between $t_1$ and $t_2$. Frequency increase is not a necessary condition for meaning shift to take place; however, it is a reasonable starting point, as a random selection of words would contain very few positive examples. Our dataset is thus biased towards precision. 
This procedure yields $\sim$200 words. We then manually identified cases of semantic shift among these words, i.e., changes in the ontological type of what a word denotes (see examples in Section~\ref{sec:types}) by analyzing their contexts of use in the r/LiverpoolFC data. This resulted in 34 words. We added two types of confounders: 33 words with a significant increase in frequency but not marked as meaning shift candidates and 33 words with constant frequency between $t_1$ and $t_2$, included as a sanity check.\footnote{All words have absolute frequency in range [50--500].} 

%We then created an online survey, which we posted in the r/LiverpoolFC to recruit participants. 
The participants were shown the 100 words (in randomized order) together with randomly chosen contexts of usage from each time period ($\mu$=4.7 contexts per word) and, for simplicity, were asked to make a binary decision
about whether there was a change in meaning. However, semantic shift is arguably a graded notion, so we aggregate the annotations into a graded \emph{semantic shift index}, ranging from 0 (no shift) to 1 (shift) depending on how many subjects spotted semantic change.\footnote{The shift index is exclusively based on the judgements by the redditors, and does not consider the preliminary candidates selection done by us.}  Overall, 26 members of r/LiverpoolFC participated in the survey, and each word received on average 8.8 judgements. The final dataset includes 97 words.\footnote{Three words were discarded: `discord'  and `owls' due to the homonymy with proper names not detected during survey's implementation; `tracking' because the chosen examples clearly mislead the judgements of the redditors.} 
Further details are in the supplementary material.


%*** SHORT VERSION***

%\paragraph{Evaluation dataset.}  
%The dataset includes 97 words, which were annotated by 26 members of the r/LiverpoolFC
%subreddit, that is, community members with domain knowledge (needed for this task) but no linguistic background.
%For each word, subjects
%read contexts of use from the two time bins and, for simplicity, were
%asked to make a binary decision
%about whether there was a change in meaning. However, semantic shift
%is arguably a graded notion, so we aggregate the annotations into a
%graded \emph{semantic shift index}, ranging from 0
%(no shift) to 1 (shift) depending on how many subjects spotted
%semantic change. This index is shown in the y-axis of
%Figure~\ref{fig:shift-cosine}. Further methodological details regarding data
%collection are described in the supplementary material. 



%*** LONG VERSION ***

%\paragraph{Evaluation dataset.}\todo{M: I commented out the short version of the paragraph and put back the long/original one, with small additions - in red} 
%For evaluation and analysis, we create a small dataset of words to be annotated as positive or negative meaning shift examples by community members without linguistic background.\footnote{Domain knowledge is needed for this task.}
%We initially leverage information about increase in frequency, which has been shown to positively correlate with meaning change \cite{wijaya2011understanding,kulkarni2015statistically}, and sample words with a significant increase in relative frequency between $t_1$ and $t_2$ (an increase is considered significant if it is at least two standard deviations above the mean).\footnote{We consider   content words only, which we identify by using the external list of common words available at \url{https://www.wordfrequency.info/free.asp}} Frequency increase is not a necessary condition for meaning shift to take place; however, given the positive correlation mentioned above, it is a reasonable starting point, as a random selection of words would contain very few positive examples. Our dataset is thus biased towards precision. 
%This procedure yields $\sim$200 words. The first author of the paper went through the list of words to identify cases of potential meaning shift, based on the analysis of the contexts of use in the r/LiverpoolFC data. By semantic shift, we mean change in the ontological type of what a word denotes (see examples in Section \ref{sect:results}). We considered only new senses - i.e., not existing senses which increase in frequency- , both for monosemous and polysemous words. This resulted in 34 words. We added two types of confounders: 33 words with a significant increase in frequency but not marked as meaning shift candidates and 33 words with constant frequency between $t_1$ and $t_2$, included as a sanity check.\footnote{All words have absolute frequency in range [50--500].} 
%
%We then created an online survey, which we posted in the r/LiverpoolFC to recruit participants. The participants were shown the 100 words together with randomly chosen contexts of usage from each time period (1 to 5 contexts depending on the word - \textcolor{red}{mean number of contexts per word=4.7})). For feasibility, they were asked to label words as `shift' or `no shift', although semantic shift is better viewed as graded (see below). The order of presentation was randomized for each participant.\footnote{Survey's instructions are in the supplementary material.} Overall, 26 members of r/LiverpoolFC participated in the survey, and each word in the dataset received on average 8.8 judgements. The final dataset includes 97 words.\footnote{Three words were discarded: `discord'  and `owls' due to the homonymy with proper names not detected during survey's implementation; `tracking' because the chosen examples clearly mislead the judgements of the redditors.} Inter-annotator agreement, computed as Krippendorff's alpha, is $\alpha$ = 0.58, a relatively low value but common in semantic tasks \cite{artstein2008inter}.
%We use the annotations to define a gradable \emph{semantic shift index}, computed as the proportion of `shift' judgements a word received in the survey.\footnote{The shift index is exclusively based on the judgements by the redditors, and does not consider the preliminary candidates selection done by one of us.} The index ranges from 0 (no shift) to 1 (shift). \textcolor{red}{As expected, all words with no frequency increase in $t_2$ have a shift index lower than 0.5 (average=0.07 $\pm$ 0.12), which validates our data selection method. For words labeled by the first author as potential shift, the average shift index is 0.72 $\pm$ 0.15, for words which increase in frequency 0.15 $\pm$ 0.16.}
%
%%% Local Variables:
%%% mode: latex
%%% TeX-master: "main"
%%% End:


%============================
\section{Types of Meaning Shift}
\label{sec:types}
We identify three main types of shift in our data via qualitative analysis of examples with a shift index $> 0.5$: metonymy, metaphor, and meme.  

In metonymic shifts, a highly salient characteristic of an entity is used to refer to it. Among these cases are, for example, {\em `highlighter'}, which in $t_2$ occurs in sentences like \textit{`we are playing with the highlighter today'},
\marco{or \textit{`what's up with the hate for this kit? This is great, ten times better than the highlighter'}}, 
used to talk about a kit in a colour similar to that of a highlighter pen; or {\em `lean'}, in \textit{`I hope a lean comes soon!'},
\marco{and \textit{`Somebody with speed...make a signing... Cuz I need a lean'}}, 
used to talk about hiring players due to new hires typically leaning on a Liverpool symbol when posing for a photo right after signing for the club. Particularly illustrative is the `F5' example shown in
Table~\ref{table:f5}. While `F5' is initially used with its common usage of shortcut for refreshing a page (1), it then starts to denote the act of refreshing in order to get the latest news about the possible transfer of a new player to LiverpoolFC (2). This use catches on and many redditors use it to express their tension while waiting for good news (3-5),\footnote{Here the semantic change is accompanied by a change in the part of speech, and `F5' becomes a denominal verb.}
though not all members are aware of the new meaning of the word (6). After the player finally signed for the team, someone leaves the {\em `F5 squad'} (7), and after a while, another member recalls the period in which the word was used (8).


\begin{table}[t]\centering   \small
    \begin{tabular}{@{}cp{5.5cm}r@{}}
        \hline
        (1) & \em after losing the F5 key on my keyboard... & 18 Jun\\\hline
        (2) & \em F5 tapping is so intense now. I want him & 28 Jun\\\hline
        (3) & \em Don't think about it too much, man. Just F5 & 1 Jul\\\hline
        (4) & \em just woke up and thought it was f5 time & 3 Jul\\\hline
        (5) & \em this was a happy f5 & 13 Jul\\\hline
        (6) & \em what is an F5? & 13 Jul \\\hline
        %[...] Nope. I didn't even know F5 was the refresh shortcut until this sub\\\hline
        (7) & \em I'm leaving the f5 squad for now & 5 Aug\\\hline
        (8) & \em I made this during the f5 madness in the sub & 6 Sep\\\hline      
    \end{tabular}
%    \vspace*{-0.2cm}
    \caption{Examples of use of `F5' with time stamps, which illustrate the speed of the meaning shift process. All the examples are from LiverpoolFC$_{17}$.}
 %   \vspace*{-0.2cm}
     \label{table:f5}
\end{table}


Metaphorical shifts lead to a broadening of the original meaning of a
word through analogy.
%\todo{please check for correctness} 
%raq: there aren't clear definitions; mentioning analogical processes seems correct
For example, in $t_2$ {\em `shovel'} occurs in sentences such as \textit{`welcome aboard, here is your shovel'} or 
\textit{`you boys know how to shovel coal'}: the team is seen as a train that is running through the season, and every supporter is asked to figuratively contribute by shoving coal into the train boiler. 

%Finally, memes---where fans use a word to make jokes and be
%sarcastic---are another prominent source of meaning shift. For
%instance, as the club was about to sign a new player named Van Dijk,
%redditors exploited the homography of the surname with the common noun
%{\em `van'} and the shoe brand `Vans': `\textit{Rumour has it Van Djik
%  was wearing these vans in the van}'. Jokes of this kind are
%positively received by the community (`\textit{Hahah I love
%  it. Anything with vans is instant karma!}') and quickly become
%frequent in it.

\marco{Finally, memes are another prominent source of meaning shift. 
In this case, fans use a word to make jokes and be sarcastic, and the new usage quickly spreads within the community. 
As an example: while Liverpool were about to sign a new player named Van Dijk, redditors started to play with the omography of the first part of the surname with the common noun `van', its plural form `vans' and the shoes brand `Vans', like in `\textit{Rumour has it Van Djik was wearing these vans in the van}' or `\textit{How many vans could Van Dijk vear if Van Dijk could vear vans}'. 
Jokes of this kind are positively received by the community (`\textit{Hahah I love it. Anything with vans is instant karma!}') and quickly become frequent in it.}



%% PREVIOUS VERSION

%We identify two main sources of shift in our data. \gbt{To do: which
%  are common to long-term / known, which are new / particular to short-term?} \marco{I am not able to say right now. I guess fig lang related ones are in common, while memes ones are not. However, I have not enough knowledge of the literature to back my guess up.}The first are
%analogical processes related to figurative language, in particular
%metaphor and metonymy.
%\gbt{to do: break down following into metaphor - explanation - example(s), then
%  metonymy - explanation - example(s)}
%  \marco{I see your point, but I am bit in trouble when I need to assign a label: for example, is `highlighter' an example of metaphor or metonymy? Maybe we could just talk about fig lang} in particular metaphoric usage of words - which leads to broadening or loosening of the original meaning of the word, and metonymy - when a highly salient characteristic of an entity is used to refer to it as a whole. Among these cases are, for example, `highlighter', which occurs in sentences like \textit{`we are playing with the \textbf{highlighter} today'} or \textit{`what's up with the hate for this kit? This is great, ten times better than the \textbf{highlighter}'}, used to talk about a kit in a colour similar to that of a highlighter; or `lean', in \textit{`I hope a \textbf{lean} comes soon!'} and \textit{`Somebody with speed...make a signing... Cuz I need a \textbf{lean}'}, due to players typically leaning on a Liverpool symbol when posing for a photo right after signing for the club. Particularly explanatory is the `F5' example shown (in chronological order) in
%Table~\ref{table:f5}. While `F5' is initially used with its common usage of shortcut for refreshing a page (line 1), it then starts to denote the act of refreshing in order to get the latest news about the possible transfer of a new player to LiverpoolFC (2). This use catches on and many redditors use it to express their tension while waiting for good news (3-5),\footnote{Here the semantic change is accompanied by a change in the part of speech, and `F5' becomes a denominal verb.}
%though not all members are aware of the new meaning of the word (6). After the player finally signed for the team, someone leaves the `F5 squad' (7), and after a while, another member recalls the period in which the word was used (8).
%
%The second main source are memes. In this case, fans start to use a
%word to make jokes or as a motto, and the new usage quickly spreads
%within the community. It is the case of `monitoring': after Liverpool
%monitored some promising players for some months without in the end
%signing any of them, fans started to use the word in fixed sentences
%(`\textit{\textbf{Monitoring} intensifies!}') which are received as sarcastic by the
%community, or to make jokes like in `\textit{I've switched my focus to
%\textbf{monitoring} Jessica Alba}'\gbt{To do: choose clearer example (I don't
%  know who J. Alba is...)} \marco{mmm, I think the example is ok, she's super famous}. The meme became so pervasive that some users expressed their disappointment, like in \textit{`I'm getting tired of the "monitoring" jokes'}. 
%
%\begin{table}[t]\centering   \small
%    \begin{tabular}{cp{5.5cm}l}
%        \hline
%        1. & after losing the F5 key on my keyboard... & 18 Jun\\\hline
%        2. & F5 tapping is so intense now. I want him & 28 Jun\\\hline
%        3. & Don't think about it too much, man. Just F5 & 1 Jul\\\hline
%        4. & just woke up and thought it was f5 time & 3 Jul\\\hline
%        5. & this was a happy f5 & 13 Jul\\\hline
%        6. & what is an F5? & 13 Jul \\\hline
%        %[...] Nope. I didn't even know F5 was the refresh shortcut until this sub\\\hline
%        7. & I'm leaving the f5 squad for now & 5 Aug\\\hline
%        8. & I made this during the f5 madness in the sub & 6 Sept\\\hline
%        
%    \end{tabular}
%    \vspace*{-0.2cm}
%    \caption{\textcolor{red}{Examples of use of `F5'. On the third column is the date of the post: the short time span  from the first to the last example gives the idea of how quick is the meaning shift process considered in this work.}}
%    \vspace*{-0.2cm}
%     \label{table:f5}
%\end{table}


%%% Local Variables:
%%% mode: latex
%%% TeX-master: "main"
%%% End:


%============================
\section{Modeling Results and Analysis}
\label{sec:results}
%\raq{I would remove the first 2 paragraphs}
%\todo[inline]{point 1}
%Our hypothesis, common to previous work, is that the meaning shift of a word is mirrored by a change in the context of usage and this, in turn, in the increased cosine distance between its time-related vector representations. The results of our experiment confirm this assumption, but they also show that this relationship between meaning shift and context change does not always hold: a change in context not always indicates a shift in meaning, and, on the other hand, a shift in meaning does not necessarily imply a change in context of use.
%\todo[inline]{point 2}

%The general hypothesis is confirmed by the positive correlation (Pearson's $r$= 0.49, $p < 0.001$) existing between cosine distance and semantic shift (see Section \ref{sec:setup}). Such a tendency can be observed in the plot in figure \ref{fig:shift-cosine}. We note that all the words that do not increase in frequency have very low shift index and low cosine distance, meaning that the model reliably identify them as cases of \textit{no} meaning shift. Following the regression line, the parallel increase of shift index and cosine distance confirms the basic assumption whereby meaning shift is reflected in context change. 

Our initial hypothesis, common to previous work, is that meaning shift
is mirrored by a change in context of usage, which should be captured
by an increased in cosine distance between the time-related vector
representations of a word. The results of our experiment confirm this
hypothesis: We find a positive correlation between cosine distance and
semantic shift in our dataset (Pearson's $r$= 0.49, $p<0.001$) - see 
Figure \ref{fig:shift-cosine}. \textcolor{red}{This result indicates that the model is indeed able to capture the majority of the cases of semantic shift identified by the authors (see Section \ref{sec:sources}).}

\begin{figure}[t]\centering
%\includegraphics[width=\columnwidth]{graph_and_csv/plots/shift-cosine-correlation-plot.png}
\includegraphics[width=\columnwidth]{images/cosine_distance_shift_index_annotated.png}
\caption{Semantic shift index vs.~cosine distance for all words in the evaluation dataset (Pearson's $r$ = 0.49, $p< 0.001$). 
Red ellipsis indicates false positives, blue ellipsis false negatives.
\label{fig:shift-cosine}}
\end{figure}

%\raq{Here I would put the highlighter and lean examples; and I would move F5 to supplementary}

%\todo{R suggest to move f5 to supplementary. G says no, it's too nice.}
%\raq{We could move this example to the supplementary material. It's very nice, but it takes a lot of space. We could just say, further examples are availble in the SM}
%\raq{We observe a negative correlation between cosine similarity and semantic shift index (Pearson's $r$= -0.49, $p < 0.001$) --- so overall the model does capture the general tendency.}
%\raq{Words with no increase in frequency all have high similarity and a shift index below 0.5  - the model identifies them reliably -- we leave them aside.}

%\raq{Then, I would introduce FP and FP:}

Although the general tendency is in line with our expectations, we
also find systematic deviations. First, \textit{false positives}, that
is, words that do not undergo semantic shift despite showing
relatively large differences in context between $t_1$ and $t_2$ (red ellipsis in
Figure~\ref{fig:shift-cosine}; shift index=0, cosine distance\textgreater	
0.25). 
%\todo{G Redraw red ellipsis to cover fewer datapoints -- maybe  higher than 0.25? Else too close to the 0 x-axis.}
Manual inspection reveals that most of these ``errors'' 
%\todo{Add  number (X out of Y)?} 
are due to a referential effect: words are used
% \paragraph{False Positives.}
% The analysis of words in this group reveals that, indeed, they occur
% in a different context in $t$ 2 compared to  $t$ 1: however, this not
% due to a meaning shift, but rather to the fact that they are used 
almost exclusively to refer to a specific person or event, and
so the context of use is narrowed down with respect to $t_1$.
%\todo{M:this is dangerous, cause we do not prove this} 
For instance, `stubborn' is
almost always used to talk about the coach of the team, who was not
there in 2013 but only in 2017; 
%`village', for the village in Africa where one of the new players comes from; 
`entourage', for the entourage of one of the stars of the team; `independence' for the
political events of Catalunya. 
%\todo{do we want to include the example  of `parked'?}. 
% the target word is temporary
% narrowing down its context of use to refer to a specific referent, but
% without a semantic shift.
In all these cases, the meaning of the word stays the same, despite the change in context. In line with the Distributional Hypothesis, the model spots the change, but it is not sensitive to its nature.
This is not a problem for long-term shift studies, because embeddings
are built on a much larger number of occurrences
% over a long period of time, 
and this makes them less sensitive to changes of referential
nature like the one presented here. However, with smaller, community corpora, this problem clearly emerges.

%\raq{To structure this, now I would add two subheadings (with paragraph command) for false positive and false negatives and put the explanations/examples you have below for each of them.}



%\begin{table}[t]\centering   \small
%    \begin{tabular}{cp{5.5cm}l}
%        \hline
%        1. & after losing the F5 key on my keyboard... & 18 Jun\\\hline
%        2. & F5 tapping is so intense now. I want him & 28 Jun\\\hline
%        3. & Don't think about it too much, man. Just F5 & 1 Jul\\\hline
%        4. & just woke up and thought it was f5 time & 3 Jul\\\hline
%        5. & this was a happy f5 & 13 Jul\\\hline
%        6. & what is an F5? & 13 Jul \\\hline
%        %[...] Nope. I didn't even know F5 was the refresh shortcut until this sub\\\hline
%        7. & I'm leaving the f5 squad for now & 5 Aug\\\hline
%        8. & I made this during the f5 madness in the sub & 6 Sept\\\hline
%        
%    \end{tabular}
%    \vspace*{-0.2cm}
%    \caption{\textcolor{red}{Examples of use of `F5'. On the third column is the date of the post: the short time span  from the first to the last example gives the idea of how quick is the meaning shift process considered in this work.}}
%    \vspace*{-0.2cm}
%     \label{table:f5}
%\end{table}

A smaller, but consistent group is that of \textit{false negatives}, 
words that undergo semantic shift but are not captured by the model
(blue ellipsis; shift index\textgreater 0.5, cosine distance\textless 0.25).
%: `dilly', `shovel', `shovels', `pharaoh'.\todo{M: dont't think list is necessary}
%\gbt{List the  words here, since they are only 4.}
%\paragraph{False Negatives.}
%On the top left of the plot cluster cases that can be considered as
%\textit{false negatives}, i.e. words that are considered cases of
%semantic shift by the redditors (high semantic shift index values),
%but not by the model (low cosine distance). 
These are cases of \textit{extended}
metaphor \cite{werth1994extended}, that is, cases in which the metaphor is 
%not limited to a single word, but it is 
developed throughout the whole text produced by an author. Also in this case, the model ``is right'', in the sense that indeed
the local context of the target words does not change in $t_2$.
 % As a result, the context taken into consideration by the model is indeed the one in which the word normally occurs, and for this reason the model is not able to spot the meaning shift. 
For instance, `pharaoh' is the nickname of an Egyptian player who
joined Liverpool in 2017 and is used in
sentences like \textit{`approved by our new \textbf{Pharaoh}
  Tutankhamun'}, \textit{'our dear Egyptian \textbf{Pharaoh},
  let's hope he becomes a God'}, and so on. Similarly, `shovel',  occurs in sentences
like \textit{`welcome aboard, here is your \textbf{shovel}'},
\textit{`you boys know how to \textbf{shovel} coal'}: the team is seen as a train that is running through the season, and every supporter is asked to give its contribution, depicted as the act of shoving coal into the train boiler. Despite the metaphoric usage, the local context of these words is similar to the literal one, and so the model does not spot the meaning shift. We expect this to happen in long-term shift models, too, but we are not aware of results confirming this.

%A similar case is `shovel' (and `shovels'), that occurs in sentences
%like \textit{`welcome aboard, here is your \textbf{shovel}'},
%\textit{`you boys know how to \textbf{shovel} coal'}, \textit{`shut up
%  and \textbf{shovel}'}. The team is seen as a train that is running fast
%through the season, and every supporter is asked to give its
%contribution, figured as the act of shoving coal into the train
%boiler. 

\paragraph{Contextual variability.}

%As we have seen, the main issue for distributional models when dealing
%with short-term shift is the interference of referential aspects in
%the detection of contextual changes. 

\textcolor{red}{From the results analysis it emerges that} the main issue for distributional models when dealing
with short-term shift is contextual change due to referential aspects \textcolor{red}{(\textit{false positive})}.
We expect that in referential cases the context of use will be \textit{narrower} than for words with actual semantic shift, because they are specific to one person or event. Hence, using a measure of \textit{contextual variability} should help spot false positives. \textcolor{red}{Here we test this hypothesis}. 
We define contextual variability as follows: for a target word, we create a vector for each of its contexts in $t_2$ by averaging the embeddings of the words  occurring in it, and define variability as the average pairwise cosine distance between context vectors.\footnote{We consider the context as the five words occurring on the left and on the right of the target word.}We then test whether contextual variability has explanatory power over cosine distance by fitting a linear regression model with these two variables as predictors and semantic shift index as dependent variable.\footnote{Contextual variability and cosine distance are not correlated in our data (Pearson's $r$= 0.18,$p> 0.05$). }
The results indicate that these two aspects are indeed complementary (contextual variability: $\beta$= 0.47, $p< 0.001$, cosine distance: $\beta$= 0.40, $p< 0.001$, adjusted $R^2$=0.44). While both shift words and referential cases change context of use in $t_2$, context variability captures the fact that only in referential cases words occur in a restricted set of contexts. The scatterplot \ref{fig:shift-variability} shows this effect visually. In future work, we plan to investigate more in depth the interplay between variability, cosine and semantic shift, both in short- and long-term meaning shift.

\begin{figure}[t]\centering
%\includegraphics[width=\columnwidth]{graph_and_csv/plots/shift-cosine-correlation-plot.png}
\includegraphics[width=\columnwidth]{images/contextual_variability_shift_index_annotated.png}
\caption{Semantic shift index vs.~context variability for all words in the evaluation dataset (Pearson's $r$ = 0.55, $p< 0.001$). Red ellipses indicates the referential cases which are incorrectly assigned high cosine distance values (false positives in the paper).
\label{fig:shift-variability}}
\end{figure}


%----
%
%As we have seen, the main issue for distributional models when dealing
%with short-term shift is contextual change due to referential cases. 
%We expect that in these cases the contexts of use will be \textit{narrower} than in the case of semantic shift, because they are specific to one person or
%event. Hence, using a measure of \textit{contextual variability} should
%help spot false positives.
%% \gbt{Explain how   contextual variability is measured; results with both (r square), as   opposed to one factor (see raquel's comment in the file).}  \gbt{I   don't think we have space to discuss this further.} 
%
%We define contextual variability as follows: for a target word, we create a vector of each of its contexts in $t_2$ by averaging the embeddings of the words which occur in it, and define variability as the average pairwise cosine distance between context vectors.\footnote{We consider the context as the five words occurring on the left and on the right of the target word.} 
%We then fit a linear regression model to predict shift index based on cosine distance and contextual variability. 
%
%The results confirm that considering contextual variability as a predictor improves prediction of shift index (contextual variability: $\beta$= .47, $p< 0.001$, cosine distance: $\beta$= .40, $p< 0.001$, Adjusted $R^2$=0.44). In particular, while meaning shifts words and referential cases have in common a change of context of use between $t$ 1 and $t$ 2, context variability captures the fact that only in referential cases words occur in a restrict set of contexts. The scatterplot in the supplementary material shows this effect visually. In future work, we plan to investigate more in depth the role of of variability and its interplay with cosine and shift, both in short and long semantic shift.



%The analysis of the results confirm our initial intuition: while meaning shifts words and false positive have in common a change of context of use between $t$ 1 and $t$ 2, the context variability index computed on $t$ 2 captures the fact that only in the latter case words occur in a restrict set of contexts. 




%============================
\section{Conclusion}
\label{sect:conc}

The goal of this preliminary study was to bring to the attention of the NLP community short-term meaning shift, an under-researched problem in the field. 
% gbt: in a short paper we don't need a recap
% Our contribution is twofold: (i) we create and make available a dataset for short-term meaning shift manually annotated by experts; (ii) we test the performance of a standard model for diachronic meaning shift on such a dataset, providing a detailed error analysis that provides insights into the phenomenon.
We hope that it will spark further research into a phenomenon which, besides being of theoretical interest, has potential practical implications for NLP downstream tasks concerned with user-generated language.
% gbt: why did you remove this last sentence? Without it it sounds too vague to me
% as modeling how words meaning rapidly change in communities would allow to better understand what their members say.






\section*{Acknowledgements}
The research carried out by the Amsterdam section of the team was partially funded by the Netherlands Organisation for Scientific Research (NWO) under VIDI grant no.~276-89-008, {\em Asymmetry in Conversation}.
This project has received funding from the European Research Council (ERC) under the European Union’s Horizon 2020 research and innovation programme (grant agreement No 715154), and from the Spanish Ram\'on y Cajal programme (grant RYC-2015-18907). This paper reflects the authors' view only, and the EU is not responsible for any use that may be made of the information it contains.
\begin{flushright}
\includegraphics[width=0.8cm]{flag_yellow_low.jpeg}  
\includegraphics[width=0.8cm]{LOGO-ERC.jpg} 
\end{flushright}

\bibliography{naaclhlt2019}
\bibliographystyle{acl_natbib}

\appendix

%\section{Further details on Data and Model}
%\label{sec:further_details_data_model}
%
%We downloaded Reddit data using the Python package Praw: \url{https://pypi.python.org/pypi/praw/}.
%
%The model was implemented using the Python package Gensim: \url{https://pypi.python.org/pypi/gensim/}.


\section{Further details on Evaluation Dataset}
\label{sec:further_details_data_model}

For our experiment, we considered content words only, which we identified by using the external list of common words available at \url{https://www.wordfrequency.info/free.asp}.

Three words were discarded from the initial list after analysis of the redditor data: `discord' and `owls' due to the homonymy with proper names not detected during survey's implementation; `tracking' because the chosen examples clearly mislead the judgements of the redditors.

As detailed in the main paper, 26 members of r/LiverpoolFC participated in the survey, and each word received on average 8.8 judgements. We computed inter-annotator agreement as Krippendorff's alpha, and obtained $\alpha$ = 0.58, a relatively low value but common in semantic tasks \cite{artstein2008inter}.

The results of the annotation validate our initial word sampling procedure:

\begin{itemize}
\item  the words that present a significant increase in frequency and were annotated as
meaning shift by us received an average shift annotation of 0.72 ($\pm$ 0.15);
\item the words that present a
significant increase in frequency but that were \emph{not} annotated as
meaning shift by us received an average shift annotation of  0.15 ($\pm$ 0.16);
\item the words that keep a constant frequency between $t_1$ and $t_2$, and we don't consider examples of meaning shift, got 0.07 ($\pm$ 0.12).
\end{itemize}

\end{document}

%%% Local Variables:
%%% mode: latex
%%% TeX-master: t
%%% End:

