Semantic change has received increasing attention in empirical Computational Linguistics / NLP in the last few years \cite{tang2018state,KutuzovEtal-coling2018}. Almost all studies so far have focused on meaning shift in long periods of time --decades to centuries. However, the genesis of meaning shift and the mechanisms that produce it operate at much shorter time spans, ranging
from the online agreement on words meaning in dyadic interactions \cite{brennan1996conceptual} to the rapid spread of new meanings in relatively small communities of people \cite{wenger1998communities,eckert-mcconnellginet1992}.
This paper is, to the best of our knowledge, the first exploration of the latter phenomenon---which we call \textit{short-term meaning shift}---using distributional representations.

Short-term shift is usually hard to observe in standard language, such
as the language of books or news, which has been the focus of
long-term studies \cite[e.g.,][]{hamilton2016diachronic,kulkarni2015statistically}, since
it takes a long time for a new meaning to be widely accepted in the standard language. 
We therefore focus on the language produced in an online community of speakers, in which the 
adoption of new meanings happens at a much faster pace \cite{Clark96,hasan2009}.

\new{Our contribution is twofold. First, we create a small dataset of short-term shift for analysis and evaluation.\footnote{Data and code will be made available upon
publication.} This is necessary because, unlike studies of long-term shift, we cannot rely on material previously gathered by linguists or lexicographers.
% A qualitative analysis suggests that metonymy, metaphor, and memes are the main sources of shift in our data.
Second, we test the behavior of a standard distributional model of semantic change when
 applied to short-term shift. 
% This kind of model is based on the hypothesis that a change in context of use mirrors a change in meaning.
Our results show that this model successfully detects most shifts in our data, but that it overgeneralizes. Specifically, the model gets confused with contextual changes due to speakers in the community often talking about particular people and events ---something intrinsic to short-term shift. We propose to use a measure of contextual variability to remedy this and showcase its potential to spot false positives of referential nature like these.}

\new{We thus make progress in understanding the nature of semantic shift and towards improving computational models thereof.}

% \marco{but
% also that, on the one hand, a change in context of use - as measured
% by cosine distance between the vector representations for a word at different periods of time
% does not always correspond to a meaning shift and, on the other hand, a meaning shift can take place also without a change in context of use. These results challenge the Distributional Hypothesis itself, which is at the basis of distributional / word embedding approaches to language in Computational Linguistics.}\todo{note: R thinks this paragraph should be the last cause it is rhetorically more powerful than the one below.}

% \marco{Finally, we investigate the role of \textit{contextual variability}, i.e. the extent to which a word tends to occur in a large set of contexts, and show that it provides valuable information which is complementary to the one about cosine distance for the detection of meaning shift on the short term.} 

%but also overgeneralises to words that exhibit an increase in frequency but no meaning shift. This challenges the Distributional Hypothesis itself, which is at the basis of distributional / word embedding approaches to language in Computational Linguistics.

%\marco{moved the paragraph to the conclusion.}
%While preliminary, our investigation opens the way to a new line of
%research within diachronic studies in our field, focusing on
%Short-term Meaning Shift originated and spread in small communities
%of speakers. Besides being of intrinsic theoretical interest, such a
%phenomenon has practical implications for every NLP downstream task
%concerning online communities, as modeling how words change their
%meaning in each community should allow a better understanding of the
%contents produced by its members.

%we study something new: meaning shift in a (relatively) short span. Main contributions:
%manually annotated dataset and analysis of the performance of current models.

%*** Our goal is to assess the performance of NLP system for long term shift detection on a short period. 

%Things to stress: in the nlp field, large work has been carried out on long term meaning shift (actually, when talking about meaning shift, the long term one - 'gay' example - is the only one considered). However, there exists also another kind on of shift, which is on the short term. While the long term meaning shift is observed at the level of general lanaguage, meaning that the new meaning o a word is gradually spread at all the levels of a (national?) community of spekaers, short term changes can not be observed at the same level, because, by their nature, they originate and are shared by smaller communities of speakers. Only if they manage, at some point, to go beyond that community and spread in other, they can be modelled as long term shifts (in the same way of 'gay'). So, in a way, exploring short term changes means looking at the genesis of the ones investigated in the nlp field up to know.

%We need to say which are the kind of shifts observed, and for this we need some theoretical framework. In any case, we have for sure memes,  referential phenomena and some kind of metaphorical process. 

%we need to say that we want also to observe how model for long term meaning shift work on this dataset. 

%%% Local Variables:
%%% mode: plain-tex
%%% TeX-master: "main"
%%% End:
