Semantic change has received increasing attention in empirical Computational Linguistics / NLP in the last few years \cite{tang2018state,KutuzovEtal-coling2018}. Almost all studies so far have focused on meaning shift in long periods of time---decades to centuries. However, the genesis of meaning shift and the mechanisms that produce it operate at much shorter time spans, ranging
from the online agreement on words' meaning in dyadic interactions \cite{brennan1996conceptual} to the rapid spread of new meanings in relatively small communities of people in \cite{wenger1998communities,eckert-mcconnellginet1992}.
This paper is, to the best of our knowledge, the first exploration of the latter phenomenon---which we call \textbf{short-term meaning shift}---using distributional representations.

More concretely, we focus on meaning shift arising within a period of 8 years, and explore it on data from an online community of speakers, because there the 
adoption of new meanings happens at a fast pace \cite{Clark96,hasan2009}.
Indeed, short-term shift is usually hard to observe in standard language, such
as the language of books or news, which has been the focus of
long-term shift studies \cite[e.g.,][]{hamilton2016diachronic,kulkarni2015statistically}, since
it takes a long time for a new meaning to be widely accepted in the standard language. 

Our contribution is twofold. First, we create a small dataset of short-term shift for analysis and evaluation, and qualitatively analyze the types of meaning shift we find.%
\footnote{\label{note1}Data and code are available at: 
\url{https://github.com/marcodel13/Short-term-meaning-shift}.} 
This is necessary because, unlike studies of long-term shift, we cannot rely on material previously gathered by linguists or lexicographers.
Second, we test the behavior of a standard distributional model of semantic change when
 applied to short-term shift. 
Our results show that this model successfully detects most shifts in our data, but it overgeneralizes. Specifically, the model gets confused with contextual changes due to speakers in the community often talking about particular people and events, which are frequent on short time spans. 
We propose to use a measure of contextual variability to remedy this and showcase its potential to spot false positives of referential nature like these.
We thus make progress in understanding the nature of semantic shift and towards improving computational models thereof.

%%% Local Variables:
%%% mode: plain-tex
%%% TeX-master: "main"
%%% End:
