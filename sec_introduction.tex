Semantic change has received
increasing attention in empirical Computational Linguistics / NLP in
the last few years \cite{tang2018state}. Almost all studies so far have focused on meaning shift in
long periods of time --decades to centuries. However, the genesis of meaning shift and the
mechanisms that produce it operate at much shorter time spans, ranging
from the online agreement on words meaning in dyadic interactions
\cite{brennan1996conceptual} to the rapid spread of new meanings in
relatively small communities of people \cite{del2017semantic}. In this
paper we focus on this latter phenomenon, that we call \textit{short-term
meaning shift}.
% \marco{(StMS)}. 
%\raq{I would here introduced the achronym StMS. Otherwise sometimes we end up using Short-term Shift, like in next para and it's a bit messy with the capitalisation and all. Maybe same for LtMS? First occurrence above } 
%\marco{answer: I agree, I am going to introduce the acronym}
%\gbt{Acronyms is something I'm strongly against, unless they are for very comomn things like NLP. They really hamper the understanding of what you say in a paper. For us it's very clear what StMS means, but for someone reading this for the first time it's not. "Short-term meaning shift" they can understand without extra cognitive load, but "StMS" not. I also find capitalization sort of pretentious. My solution: de-capitalize everywhere and use the full form. This way we can use either short-term meaning shift, or shorten the expression to short-term shift or shift as needed in the context.}

Short-term shift is usually hard to observe in standard language, such
as the language of books or news, which has been the focus of
long-term studies \cite{hamilton2016diachronic,kulkarni2015statistically}, since
%, as is the case for any
%other linguistic innovation, 
it takes a long time for a new meaning to be widely accepted in the standard language. 
% In contrast, 
% new meanings are continuously created and shared in small
% communities of speakers, in which people involved in direct
% communication create and spread those innovations \cite{Clark96,hasan2009}.
We therefore focus on the language produced in an online community of speakers, in which the 
%genesis, spread and 
adoption of new meanings happens at a much faster pace \cite{Clark96,hasan2009}.
%\todo{How about decay? (death of new meanings) -- better not to mention for simplicity?} 
%\todo{G added these references from the previous version of this sentence, pls check they're ok here.}

We analyze the behavior of a standard distributional model of
semantic change when applied to short-term shift,
also creating a small dataset for this purpose.\footnote{Data and code
will be made available upon publication.} 
Distributional models of semantic change are based on the hypothesis that a change in context of use mirrors a change in meaning.
Our results show that this type of model successfully detects most meaning shifts, but that it overgeneralizes, since some contextual changes do not correspond to a meaning shift.
%\gbt{this is the main result; I would not mention the false negatives cause they are very few.}
We also show that this is a difficulty caused by the nature of short-term meaning shift, and propose to use contextual variability as a means to remedy it.

% \marco{but
% also that, on the one hand, a change in context of use - as measured
% by cosine distance between the vector representations for a word at different periods of time
% does not always correspond to a meaning shift and, on the other hand, a meaning shift can take place also without a change in context of use. These results challenge the Distributional Hypothesis itself, which is at the basis of distributional / word embedding approaches to language in Computational Linguistics.}\todo{note: R thinks this paragraph should be the last cause it is rhetorically more powerful than the one below.}

% \marco{Finally, we investigate the role of \textit{contextual variability}, i.e. the extent to which a word tends to occur in a large set of contexts, and show that it provides valuable information which is complementary to the one about cosine distance for the detection of meaning shift on the short term.} 

%but also overgeneralises to words that exhibit an increase in frequency but no meaning shift. This challenges the Distributional Hypothesis itself, which is at the basis of distributional / word embedding approaches to language in Computational Linguistics.

%\marco{moved the paragraph to the conclusion.}
%While preliminary, our investigation opens the way to a new line of
%research within diachronic studies in our field, focusing on
%Short-term Meaning Shift originated and spread in small communities
%of speakers. Besides being of intrinsic theoretical interest, such a
%phenomenon has practical implications for every NLP downstream task
%concerning online communities, as modeling how words change their
%meaning in each community should allow a better understanding of the
%contents produced by its members.

%we study something new: meaning shift in a (relatively) short span. Main contributions:
%manually annotated dataset and analysis of the performance of current models.

%*** Our goal is to assess the performance of NLP system for long term shift detection on a short period. 

%Things to stress: in the nlp field, large work has been carried out on long term meaning shift (actually, when talking about meaning shift, the long term one - 'gay' example - is the only one considered). However, there exists also another kind on of shift, which is on the short term. While the long term meaning shift is observed at the level of general lanaguage, meaning that the new meaning o a word is gradually spread at all the levels of a (national?) community of spekaers, short term changes can not be observed at the same level, because, by their nature, they originate and are shared by smaller communities of speakers. Only if they manage, at some point, to go beyond that community and spread in other, they can be modelled as long term shifts (in the same way of 'gay'). So, in a way, exploring short term changes means looking at the genesis of the ones investigated in the nlp field up to know.

%We need to say which are the kind of shifts observed, and for this we need some theoretical framework. In any case, we have for sure memes,  referential phenomena and some kind of metaphorical process. 

%we need to say that we want also to observe how model for long term meaning shift work on this dataset. 

%%% Local Variables:
%%% mode: plain-tex
%%% TeX-master: "main"
%%% End:
