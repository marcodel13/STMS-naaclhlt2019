We identify three main types of shift in our data via qualitative analysis of examples with a shift index $> 0.5$: metonymy, metaphor, and meme.  

\paragraph{Metonymy.}
In metonymic shifts, a highly salient characteristic of an entity is used to refer to it. Among these cases are, for example, {\em `highlighter'}, which in $t_2$ occurs in sentences like \textit{`we are playing with the highlighter today'}, or \textit{`what's up with the hate for this kit? This is great, ten times better than the highlighter'},
used to talk about a kit in a colour similar to that of a highlighter pen; or {\em `lean'}, in \textit{`I hope a lean comes soon!'}, \textit{`Somebody with speed\dots make a signing\dots Cuz I need a lean'},
used to talk about hiring players due to new hires typically leaning on a Liverpool symbol when posing for a photo right after signing for the club. Particularly illustrative is the `F5' example shown in
Table~\ref{table:f5}. While `F5' is initially used with its common usage of shortcut for refreshing a page (1), it then starts to denote the act of refreshing in order to get the latest news about the possible transfer of a new player to LiverpoolFC (2). This use catches on and many redditors use it to express their tension while waiting for good news (3-5),\footnote{Here the semantic change is accompanied by a change in the part of speech, and `F5' becomes a denominal verb.}
though not all members are aware of the new meaning of the word (6). When the transfer is almost done, someone leaves the {\em `F5 squad'} (7), and after a while, another member recalls the period in which the word was used (8).

\begin{table}[t]\centering
    \begin{tabular}{@{}c@{\ \ }p{5.5cm}r@{}}
        \hline
        (1) & \em Damn, after losing the F5 key on my keyboard [...] & 16 Jun\\\hline
        (2) & \em [he is] so  close, F5 tapping is so intense right now & 18 Jun\\\hline
        (3) & \em Don't think about it too much, man. Just F5 & 1 Jul\\\hline
        (4) & \em Literally 4am I slept and just woke up and thought it was f5 time & 3 Jul\\\hline
        (5) & \em this was a happy f5 & 3 Jul\\\hline
        (6) & \em what is an F5? & 3 Jul \\\hline
        %[...] Nope. I didn't even know F5 was the refresh shortcut until this sub\\\hline
        (7) & \em I'm leaving the f5 squad for now & 5 Jul\\\hline
        (8) & \em I made this during the f5 madness & 6 Sep\\\hline      
    \end{tabular}
    \caption{Examples of use of `F5' with time stamps, which illustrate the speed of the meaning shift process. All the examples are from LiverpoolFC$_{17}$.}
     \label{table:f5}
\end{table}

\paragraph{Metaphor.}
Metaphorical shifts lead to a broadening of the original meaning of a
word through analogy.
For example, in $t_2$ {\em `shovel'} occurs in sentences such as \textit{`welcome aboard, here is your shovel'} or 
\textit{`you boys know how to shovel coal'}: the team is seen as a train that is running through the season, and every supporter is asked to figuratively contribute by shoving coal into the train boiler. 

\paragraph{Meme.}
Finally, memes are another prominent source of meaning shift. 
In this case, fans use a word to make jokes and be sarcastic, and the new usage quickly spreads within the community. 
As an example, while Liverpool was about to sign a new player named Van Dijk, redditors started to play with the homography of the first part of the surname with the common noun `van', its plural form `vans', and the shoes brand `Vans': `\textit{Rumour has it Van Djik was wearing these vans in the van}' or `\textit{How many vans could Van Dijk vear if Van Dijk could vear vans}'. 
Jokes of this kind are positively received in the community (`\textit{Hahah I love it. Anything with vans is instant karma!}') and quickly become frequent in it.

%%% Local Variables:
%%% mode: latex
%%% TeX-master: "main"
%%% End:
