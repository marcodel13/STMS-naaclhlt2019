% we identified three main kinds of shift: (i) metaphoric usage, which leads to broadening or loosening of the original meaning of the word; (ii) metonymy, when a highly salient characteristic of an entity is used to refer to it as a whole; (iii) meme: fans start to use a word to make jokes or as a motto, and the new usage quickly spreads within the community. 

We identify two main sources of shift in our data. \gbt{To do: which
  are common to long-term / known, which are new / particular to short-term?} The first are
analogical processes related to figurative language, in particular
metaphor and metonymy.
\gbt{to do: break down following into metaphor - explanation - example(s), then
  metonymy - explanation - example(s)} in particular metaphoric usage of words - which leads to broadening or loosening of the original meaning of the word, and metonymy - when a highly salient characteristic of an entity is used to refer to it as a whole. Among these cases are, for example, `highlighter', which occurs in sentences like \textit{`we are playing with the \textbf{highlighter} today'} or \textit{`what's up with the hate for this kit? This is great, ten times better than the \textbf{highlighter}'}, used to talk about a kit in a colour similar to that of a highlighter; or `lean', in \textit{`I hope a \textbf{lean} comes soon!'} and \textit{`Somebody with speed...make a signing... Cuz I need a \textbf{lean}'}, due to players typically leaning on a Liverpool symbol when posing for a photo right after signing for the club. Particularly explanatory is the `F5' example shown (in chronological order) in
Table~\ref{table:f5}. While `F5' is initially used with its common usage of shortcut for refreshing a page (line 1), it then starts to denote the act of refreshing in order to get the latest news about the possible transfer of a new player to LiverpoolFC (2). This use catches on and many redditors use it to express their tension while waiting for good news (3-5),\footnote{Here the semantic change is accompanied by a change in the part of speech, and `F5' becomes a denominal verb.}
though not all members are aware of the new meaning of the word (6). After the player finally signed for the team, someone leaves the `F5 squad' (7), and after a while, another member recalls the period in which the word was used (8).

The second main source are memes. In this case, fans start to use a
word to make jokes or as a motto, and the new usage quickly spreads
within the community. It is the case of `monitoring': after Liverpool
monitored some promising players for some months without in the end
signing any of them, fans started to use the word in fixed sentences
(`Monitoring intensifies!') which are received as sarcastic by the
community, or to make jokes like in `I've switched my focus to
monitoring Jessica Alba'\gbt{To do: choose clearer example (I don't
  know who J. Alba is...)}. The meme became so pervasive that some users expressed their disappointment, like in \textit{`I'm getting tired of the "monitoring" jokes'}. \marco{do we want another example here?}\gbt{No}

\begin{table}[t]\centering   \small
    \begin{tabular}{cp{5.5cm}l}
        \hline
        1. & after losing the F5 key on my keyboard... & 18 Jun\\\hline
        2. & F5 tapping is so intense now. I want him & 28 Jun\\\hline
        3. & Don't think about it too much, man. Just F5 & 1 Jul\\\hline
        4. & just woke up and thought it was f5 time & 3 Jul\\\hline
        5. & this was a happy f5 & 13 Jul\\\hline
        6. & what is an F5? & 13 Jul \\\hline
        %[...] Nope. I didn't even know F5 was the refresh shortcut until this sub\\\hline
        7. & I'm leaving the f5 squad for now & 5 Aug\\\hline
        8. & I made this during the f5 madness in the sub & 6 Sept\\\hline
        
    \end{tabular}
    \vspace*{-0.2cm}
    \caption{\textcolor{red}{Examples of use of `F5'. On the third column is the date of the post: the short time span  from the first to the last example gives the idea of how quick is the meaning shift process considered in this work.}}
    \vspace*{-0.2cm}
     \label{table:f5}
\end{table}
%%% Local Variables:
%%% mode: latex
%%% TeX-master: "main"
%%% End:
