We identify three main types of shift in our data via qualitative analysis of examples with a shift index $> 0.5$: metonymy, metaphor, and meme.  

In metonymic shifts, a highly salient characteristic of an entity is used to refer to it. Among these cases are, for example, {\em `highlighter'}, which in $t_2$ occurs in sentences like \textit{`we are playing with the highlighter today'},
%or \textit{`what's up with the hate for this kit? This is great, ten times better than the highlighter'}, 
used to talk about a kit in a colour similar to that of a highlighter pen; or {\em `lean'}, in \textit{`I hope a lean comes soon!'},
%and \textit{`Somebody with speed...make a signing... Cuz I need a lean'}, 
used to talk about hiring players due to new hires typically leaning on a Liverpool symbol when posing for a photo right after signing for the club. Particularly illustrative is the `F5' example shown in
Table~\ref{table:f5}. While `F5' is initially used with its common usage of shortcut for refreshing a page (1), it then starts to denote the act of refreshing in order to get the latest news about the possible transfer of a new player to LiverpoolFC (2). This use catches on and many redditors use it to express their tension while waiting for good news (3-5),\footnote{Here the semantic change is accompanied by a change in the part of speech, and `F5' becomes a denominal verb.}
though not all members are aware of the new meaning of the word (6). After the player finally signed for the team, someone leaves the {\em `F5 squad'} (7), and after a while, another member recalls the period in which the word was used (8).


\begin{table}[t]\centering   \small
    \begin{tabular}{@{}cp{5.5cm}r@{}}
        \hline
        (1) & \em after losing the F5 key on my keyboard... & 18 Jun\\\hline
        (2) & \em F5 tapping is so intense now. I want him & 28 Jun\\\hline
        (3) & \em Don't think about it too much, man. Just F5 & 1 Jul\\\hline
        (4) & \em just woke up and thought it was f5 time & 3 Jul\\\hline
        (5) & \em this was a happy f5 & 13 Jul\\\hline
        (6) & \em what is an F5? & 13 Jul \\\hline
        %[...] Nope. I didn't even know F5 was the refresh shortcut until this sub\\\hline
        (7) & \em I'm leaving the f5 squad for now & 5 Aug\\\hline
        (8) & \em I made this during the f5 madness in the sub & 6 Sep\\\hline      
    \end{tabular}
    \vspace*{-0.2cm}
    \caption{Examples of use of `F5' with time stamps, which illustrate the speed of the meaning shift process.}
    \vspace*{-0.2cm}
     \label{table:f5}
\end{table}


Metaphorical shifts lead to a broadening of the original meaning of a
word \new{through analogy}.\todo{please check for correctness} For example, in $t_2$ {\em `shovel'} occurs in sentences such as \textit{`welcome aboard, here is your shovel'} or 
\textit{`you boys know how to shovel coal'}: the team is seen as a train that is running through the season, and every supporter is asked to figuratively contribute by shoving coal into the train boiler. 

Finally, memes---where fans use a word to make jokes and be
sarcastic---are another prominent source of meaning shift. For
instance, as the club was about to sign a new player named Van Dijk,
redditors exploited the homography of the surname with the common noun
{\em `van'} and the shoe brand `Vans': `\textit{Rumour has it Van Djik
  was wearing these vans in the van}'. Jokes of this kind are
positively received by the community (`\textit{Hahah I love
  it. Anything with vans is instant karma!}') and quickly become
frequent in it.
%\todo{Gemma says: (Idle comment:) I don't know what to make of memes,
%  from a semantic perspective.}

%Finally, memes are another prominent source of meaning shift. In this case, fans use a word to make jokes and be sarcastic, and the new usage quickly spreads within the community. As an example: while Liverpool were about to sign a new player named Van Dijk, redditors started to play with the omography of the first part of the surname with the common noun `van', its plural form `vans' and the shoes brand `Vans', like in `\textit{Rumour has it Van Djik was wearing these vans in the van}' or `\textit{How many vans could Van Dijk vear if Van Dijk could vear vans}'. Jokes of this kind are positively received by the community (`\textit{Hahah I love it. Anything with vans is instant karma!}') and quickly become frequent in it.}



%% PREVIOUS VERSION

%We identify two main sources of shift in our data. \gbt{To do: which
%  are common to long-term / known, which are new / particular to short-term?} \marco{I am not able to say right now. I guess fig lang related ones are in common, while memes ones are not. However, I have not enough knowledge of the literature to back my guess up.}The first are
%analogical processes related to figurative language, in particular
%metaphor and metonymy.
%\gbt{to do: break down following into metaphor - explanation - example(s), then
%  metonymy - explanation - example(s)}
%  \marco{I see your point, but I am bit in trouble when I need to assign a label: for example, is `highlighter' an example of metaphor or metonymy? Maybe we could just talk about fig lang} in particular metaphoric usage of words - which leads to broadening or loosening of the original meaning of the word, and metonymy - when a highly salient characteristic of an entity is used to refer to it as a whole. Among these cases are, for example, `highlighter', which occurs in sentences like \textit{`we are playing with the \textbf{highlighter} today'} or \textit{`what's up with the hate for this kit? This is great, ten times better than the \textbf{highlighter}'}, used to talk about a kit in a colour similar to that of a highlighter; or `lean', in \textit{`I hope a \textbf{lean} comes soon!'} and \textit{`Somebody with speed...make a signing... Cuz I need a \textbf{lean}'}, due to players typically leaning on a Liverpool symbol when posing for a photo right after signing for the club. Particularly explanatory is the `F5' example shown (in chronological order) in
%Table~\ref{table:f5}. While `F5' is initially used with its common usage of shortcut for refreshing a page (line 1), it then starts to denote the act of refreshing in order to get the latest news about the possible transfer of a new player to LiverpoolFC (2). This use catches on and many redditors use it to express their tension while waiting for good news (3-5),\footnote{Here the semantic change is accompanied by a change in the part of speech, and `F5' becomes a denominal verb.}
%though not all members are aware of the new meaning of the word (6). After the player finally signed for the team, someone leaves the `F5 squad' (7), and after a while, another member recalls the period in which the word was used (8).
%
%The second main source are memes. In this case, fans start to use a
%word to make jokes or as a motto, and the new usage quickly spreads
%within the community. It is the case of `monitoring': after Liverpool
%monitored some promising players for some months without in the end
%signing any of them, fans started to use the word in fixed sentences
%(`\textit{\textbf{Monitoring} intensifies!}') which are received as sarcastic by the
%community, or to make jokes like in `\textit{I've switched my focus to
%\textbf{monitoring} Jessica Alba}'\gbt{To do: choose clearer example (I don't
%  know who J. Alba is...)} \marco{mmm, I think the example is ok, she's super famous}. The meme became so pervasive that some users expressed their disappointment, like in \textit{`I'm getting tired of the "monitoring" jokes'}. 
%
%\begin{table}[t]\centering   \small
%    \begin{tabular}{cp{5.5cm}l}
%        \hline
%        1. & after losing the F5 key on my keyboard... & 18 Jun\\\hline
%        2. & F5 tapping is so intense now. I want him & 28 Jun\\\hline
%        3. & Don't think about it too much, man. Just F5 & 1 Jul\\\hline
%        4. & just woke up and thought it was f5 time & 3 Jul\\\hline
%        5. & this was a happy f5 & 13 Jul\\\hline
%        6. & what is an F5? & 13 Jul \\\hline
%        %[...] Nope. I didn't even know F5 was the refresh shortcut until this sub\\\hline
%        7. & I'm leaving the f5 squad for now & 5 Aug\\\hline
%        8. & I made this during the f5 madness in the sub & 6 Sept\\\hline
%        
%    \end{tabular}
%    \vspace*{-0.2cm}
%    \caption{\textcolor{red}{Examples of use of `F5'. On the third column is the date of the post: the short time span  from the first to the last example gives the idea of how quick is the meaning shift process considered in this work.}}
%    \vspace*{-0.2cm}
%     \label{table:f5}
%\end{table}


%%% Local Variables:
%%% mode: latex
%%% TeX-master: "main"
%%% End:
