For this study, we focus on an online forum of football fans,
namely the r/LiverpoolFC subreddit, one of the many communities hosted
by Reddit.\footnote{\url{https://www.reddit.com}.} 
%Each of these
%subreddits is essentially a forum in which subscribed users talk about
%a specific topic, submit posts and comment on existing threads. 
A community such as that of football fans tends to be very active and
 interconnected, and hence is a good environment for observing
linguistic innovations like meaning
shift~\cite{hamilton2017loyalty}.

We observe meaning shift in the period between 2013 and 2017.
Following the typical approach in diachronic studies in NLP, we split
the r/LiverpoolFC subreddit data into time bins. In order to enable a
clear observation of short-term meaning shift, we define two,
non-consecutive time bins: the first ($t_1$) contains data from
2011-2013 and the second ($t_2$) from 2017.\footnote{These choices
  ensure that the two datasets are approximately of the same size. The
  r/LiverpoolFC subreddit exists since 2009, but very little content
  was produced in 2009-2010.}
% each of which is used to create
%time-dependent word representations. We make the time bins
%non-consecutive to enable observation of short-time meaning shift (see
%Introduction). We end up with two time bins, where the first bin ($t1$) contains data from 2011-2013 and the second ($t2$)
%from 2017.\footnote{These choices ensure that the two datasets are
%  approximately of the same size. The r/LiverpoolFC subreddit exists
%  since 2009, but very little content was produced in 2009-2010.}.
 We also use a large sample of community-independent language for the
initialization of the word vectors, namely, a random crawl from Reddit
in 2013. For all three samples, we downloaded both textual content and time stamp of the posts (Table~\ref{tab:data} shows
sample sizes).\footnote{We used the Python package Praw,
  \url{https://pypi.python.org/pypi/praw}.}

%----- TABLE ----------------------------------------------------
\begin{table}[t]
\centering
\begin{tabular}{lccc}
\bf sample & \bf time bin & \bf million tokens \\
 \hline
\redd &  2013 & {\raise.17ex\hbox{$\scriptstyle\sim$}}900 \\
LiverpoolFC$_{13}$ & 2011-13 & 8.5\\
LiverpoolFC$_{17}$ & 2017 & 11.9\\
\end{tabular}
\caption{Time bin and size of the datasets.}
\label{tab:data}
\end{table}
%----- END TABLE -----------------------------------------

\paragraph{Dataset.} For evaluation and analysis, we create a dataset
with positive and negative meaning shift examples based on the
linguistic material produced in r/LiverpoolFC.

% A random sample of words would not contain many positive cases of
% semantic, since most of the words in language tend to maintain a
% relatively similar meaning in time.
In order to create the dataset starting from the raw text downloaded
from the subreddit, we initially leverage information about increase
in frequency, which has been shown to positively correlate with
meaning change
\cite{wijaya2011understanding,kulkarni2015statistically}, and sample
words with a significant increase in frequency between $t_1$ and $t_2$
(an increase in relative frequency is considered significant if it is
at least two standard deviations above the mean).\footnote{We consider
  content words only, which we identify by using the external list
  of common words available at
  \url{https://www.wordfrequency.info/free.asp}} Frequency increase is
not a necessary condition for meaning shift to take place; however,
given the positive correlation mentioned above, it is a reasonable
start, since a random selection of words would contain very few
positive examples.

This procedure yields {\raise.17ex\hbox{$\scriptstyle\sim$}}200 words.
The first author of the paper went through the list of words to
identify cases of meaning shift, based on the analysis of the contexts
of use in the r/LiverpoolFC corpus. We then collected a set of positive
and negative examples (100 words in total), which includes 1) words
that present a significant increase in frequency and were annotated as
meaning shift by the first author (34 words), 2) words that present a
significant increase in frequency but that were not annotated as
meaning shift by the first author (33 words), and 3) words that keep a
constant frequency between $t_1$ and $t_2$, and are not considered as
examples of meaning shift by us (33 words).\footnote{Words in the three lists have overall absolute frequency included in range [50-500].}

We then created the final dataset via an online survey, which we posted
in the r/LiverpoolFC subreddit to recruit participants.\footnote{Domain knowledge is needed for this task.} 
%with some of the members of the r/LiverpoolFC subreddit.\footnote{Domain knowledge is needed for this task.} 
We provided participants with the 100 words
together with randomly chosen examples from each time period (1 to 5
examples depending on the word). They were asked to label as many of
the words as they wanted as `shift' or `no shift'. The order of presentation
was randomized for each participant.

Overall, 26 members of r/LiverpoolFC participated in the survey, and
each word in the dataset received between 5 and 12 judgements
(average=8.8). The interannotator agreement, computed as
Krippendorff's alpha, is
0.58 --a moderate level of
agreement, as is common for semantic tasks \cite{artstein2008inter}. We assigned a
final label to each word in the dataset according to the majority of
votes received by the redditors. This resulted in a final dataset
including 21 cases of meaning shift and 76 of no shift (total 97
words).\footnote{Three words were removed from the dataset: `discord'
  and `owls' due to the homonymy with proper names not detected by the
  authors during the implementation of the survey; `tracking' because
  the chosen examples clearly mislead the judgements of the
  redditors.}

%
%%----- TABLE ----------------------------------------------------
%\begin{table*}[t]
%\centering
%\begin{tabular}{lc}
%\bf target word & \bf usage \\
% \hline
%word 1 & text texttexttexttexttexttexttexttexttexttext\\
%word 2 & texttexttexttexttexttexttexttexttexttext\\
%\end{tabular}
%\caption{Time bin and size of the datasets.}
%\label{tab:data}
%\end{table*}
%%----- END TABLE -----------------------------------------

% The analysis of the words tagged as having undergone a shift in meaning reveals some clear types of semantic processes, such as:
% \begin{itemize}[leftmargin=12pt,itemsep=0pt]
% \item Figurative modulations such as metonymy (e.g.,
%   \emph{highlighter} for a sports kit in a colour similar to that of a
%   highlighter) and broadening (e.g., \emph{to lean} for `to sign up
%   for a club', due to players typically leaning on something when
%   posing for a photo right after signing in for the club).
% \item Meaning conventions\todo{Removed referential in order to avoid confusion with sec 5}: conceptual pacts
%   \cite{brennan1996conceptual} on how to refer to an entity or group
%   (e.g., \emph{madman} to refer to the coach or \emph{believers}
%   to refer to the LiverpoolFC fans) that have become conventionalised
%   within the community.
% \item Memes: mottos or slogans repeated for humorous purposes or
%   community cohesiveness (e.g., \emph{coal} in expressions such as
%   `more coal' uttered when cheering)
% \end{itemize}
%\todo{Removing ling analysis to make room for examples -- that is more important.}